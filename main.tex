\documentclass{article}
\usepackage[spanish,english,portuguese]{babel}

% Set page size and margins
% Replace `letterpaper' with `a4paper' for UK/EU standard size
\usepackage[a4paper,top=2cm,bottom=2cm,left=3cm,right=3cm,marginparwidth=1.75cm]{geometry}
\usepackage{amsmath}
\usepackage{graphicx}
\usepackage[colorlinks=true, allcolors=blue]{hyperref}
\usepackage[utf8]{inputenc}
\usepackage{csquotes}
\usepackage{natbib}
\usepackage{url}
\usepackage{hyperref}
\providecommand{\keywords}[1]
{
  \small	
  \textbf{\textit{Keywords---}} #1
}

\title{Criterios para seleccionar repositorios de datos abiertos en investigación educativa}

% \author{Gonzalo Torterolo-Silva - ORCID https://orcid.org/0000-0002-1426-562X}
% \author{Regina Motz - ORCID https://orcid.org/0000-0002-1426-562X}

\begin{document}
\selectlanguage{spanish}
\maketitle

\selectlanguage{english}
\begin{abstract}
This article discusses the criteria that influence the selection of research data repositories by educational researchers. It highlights the growing importance of public access to data in the context of open science and examines the FAIR (Findable, Accessible, Interoperable, Reusable) principles as a guide for data management. The study is based on a mixed methodology, including a literature review, expert consultation and analysis of repository catalogs.\\

The literature review identified key factors such as repository reputation, ease of use, metadata availability, and compliance with FAIR principles. Expert consultation emphasized the importance of persistent identifiers, sustainability and interoperability of repositories. Catalog analysis revealed that FAIRSharing stands out for its curation and data quality, while re3data is a primary source of information.\\

The article concludes that educational researchers need clear tools and criteria for selecting appropriate repositories. It stresses the need for repositories that facilitate interoperability and that have institutional support and technical assistance. The study provides valuable information to promote the adoption of open data practices in the field of educational research.\\
\end{abstract}

\selectlanguage{portuguese}
\begin{abstract}
Este artigo discute os critérios que influenciam a seleção de repositórios de dados de pesquisa por pesquisadores educacionais. Ele destaca a importância crescente do acesso público aos dados no contexto da ciência aberta e examina os princípios FAIR (Findable, Accessible, Interoperable, Reusable) como um guia para o gerenciamento de dados. O estudo baseia-se em uma metodologia mista, incluindo uma revisão da literatura, uma consulta a especialistas e uma análise de catálogos de repositórios.\\

A revisão da literatura identificou fatores importantes, como a reputação do repositório, a facilidade de uso, a disponibilidade de metadados e a conformidade com os princípios FAIR. A consulta a especialistas enfatizou a importância dos identificadores persistentes, da sustentabilidade e da interoperabilidade dos repositórios. A análise dos catálogos revelou que o FAIRSharing se destaca por sua curadoria e qualidade de dados, enquanto o re3data é uma fonte primária de informações.\\

O artigo conclui que os pesquisadores educacionais precisam de ferramentas e critérios claros para selecionar os repositórios adequados. Ele destaca a necessidade de repositórios que facilitem a interoperabilidade e que tenham apoio institucional e assistência técnica. O estudo fornece informações valiosas para promover a adoção de práticas de dados abertos no campo da pesquisa educacional.\\
\end{abstract}

\selectlanguage{spanish}
\begin{abstract}
Este artículo analiza los criterios que influyen en la selección de repositorios de datos de investigación por parte de investigadores educativos. Se destaca la creciente importancia del acceso público a los datos en el contexto de la ciencia abierta y se examinan los principios FAIR (Encontrables, Accesibles, Interoperables, Reutilizables) como guía para la gestión de datos. El estudio se basa en una metodología mixta, que incluye una revisión bibliográfica, una consulta a expertos y un análisis de catálogos de repositorios.\\

La revisión bibliográfica identificó factores clave como la reputación del repositorio, la facilidad de uso, la disponibilidad de metadatos y el cumplimiento de los principios FAIR. La consulta a expertos enfatizó la importancia de los identificadores persistentes, la sostenibilidad y la interoperabilidad de los repositorios. El análisis de catálogos reveló que FAIRSharing se destaca por su curaduría y calidad de datos, mientras que re3data es una fuente primaria de información.\\

El artículo concluye que los investigadores educativos necesitan herramientas y criterios claros para seleccionar repositorios adecuados. Se subraya la necesidad de repositorios que faciliten la interoperabilidad y que cuenten con apoyo institucional y asistencia técnica. El estudio proporciona información valiosa para promover la adopción de prácticas de datos abiertos en el campo de la investigación educativa.\\

% Finalmente, se destaca el rol del investigador como agente activo que colabora en la  democratización del conocimiento científico, promoviendo una cultura de datos abiertos en el ámbito educativo.\\
\end{abstract}

\keywords{research-tools, online-catalogs, metadata, data-use, archive}


\section{Introducción}

En la era digital actual, el movimiento de ciencia abierta ha reconfigurado profundamente los paradigmas de producción y circulación del conocimiento científico. Dentro de este marco transformador, el acceso público a los datos de investigación emerge como un componente estratégico fundamental, particularmente en el campo educativo donde la transparencia metodológica y la posibilidad de reanálisis cobran especial relevancia pedagógica y social. La UNESCO \cite[p.~10]{unesco2021} conceptualiza los datos abiertos de investigación como aquellos que deben estar ``disponibles de manera oportuna, en un formato fácil de utilizar, legible y modificable por personas y máquinas'', articulándose bajo los principios FAIR (Fáciles de encontrar, Accesibles, Interoperables y Reutilizables) que han sido ampliamente adoptados por la comunidad científica internacional.\\

Esta aproximación normativa no solo robustece los pilares de transparencia y reproducibilidad que sustentan el método científico \cite{oecd2020}, sino que además facilita el análisis secundario y contribuye al impacto social del conocimiento (\cite{ec2016}; \cite{wilkinson2016}). Los beneficios potenciales son particularmente significativos en el ámbito educativo, donde la posibilidad de contrastar resultados entre diferentes contextos socioculturales y sistemas pedagógicos podría generar insights transformadores para las políticas públicas en educación.\\

Ante este panorama promisorio, organismos internacionales como la \cite{unesco2021} y la \cite{oecd2020} han impulsado políticas que fomentan el uso de repositorios digitales especializados para el almacenamiento, preservación y diseminación ética de datos científicos. Como argumentan \cite{borgman2018}, estas infraestructuras digitales han dejado de ser meros almacenes de información para convertirse en nodos fundamentales de las redes contemporáneas de producción de conocimiento, facilitando no solo el acceso sino también la interoperabilidad y el enriquecimiento progresivo de los datasets mediante sucesivas reutilizaciones. En esta misma línea, la reciente conceptualización de \cite[p.~27]{avila2024} enfatiza su rol como sistemas dinámicos que permiten ``almacenar, organizar, recuperar y acceder a conjuntos de datos de diversa naturaleza temática y tipológica'', constituyéndose así en motores esenciales para la ciencia abierta.\\

No obstante, como advierte \cite{borgman2016} en su análisis sobre las ecologías del conocimiento, los datos solo alcanzan su pleno valor científico cuando se integran en ecosistemas robustos que articulen coherentemente dimensiones técnicas, prácticas comunitarias e institucionales. Precisamente esta articulación presenta desafíos particulares en el ámbito educativo y de ciencias sociales, donde, como documentan \cite{kraehmer2023} y \cite{gray2023}, la disponibilidad de repositorios disciplinares especializados palidece en comparación con otras áreas del conocimiento. Esta disparidad resulta paradójica si consideramos la complejidad intrínseca de los datos educativos, que abarcan desde mediciones cuantitativas estandarizadas hasta registros etnográficos profundamente contextualizados, frecuentemente conteniendo información sensible sobre menores de edad en entornos escolares específicos (\cite{gomes2022}).\\

Los principios FAIR (\cite{wilkinson2016}) y TRUST (\cite{Lin2020TRUST}) constituyen marcos de referencia valiosos para evaluar la calidad de los repositorios científicos. La exhaustiva caracterización de \cite{Behnke2020FAIRfeatures} detalla cómo estos incluyen tanto requerimientos organizacionales (como políticas claras de gestión de datos y soporte para formatos estándar) como técnicos (desde metadatos granularizados hasta sistemas de identificadores persistentes). Sin embargo, su implementación en investigación educativa enfrenta obstáculos específicos, especialmente cuando se trabaja con datos cualitativos que como  destaca \cite{antonio}, demandan niveles excepcionales de contextualización y mecanismos especializados de protección ética que muchos repositorios generalistas no están preparados para ofrecer.\\

Frente a estos desafíos, iniciativas como re3data \cite{pampel2013} han buscado crear catálogos integrados que faciliten la localización y evaluación de repositorios especializados.\\

No obstante, la aplicación de estos principios enfrenta retos particulares en disciplinas como la educación y las ciencias sociales, donde la heterogeneidad, sensibilidad y complejidad de los datos dificultan la estandarización y la creación de infraestructuras especializadas. La escasez de repositorios disciplinares en estas áreas, sumada a desafíos éticos, legales y culturales, obliga a los investigadores a recurrir mayoritariamente a repositorios institucionales o generalistas, que no siempre satisfacen los requerimientos específicos para la gestión óptima de datos educativos y sociales.\\

Frente a este panorama, la selección informada de repositorios se vuelve un desafío crucial. Los catálogos de repositorios, como re3data, emergen como herramientas clave para buscar, filtrar y evaluar opciones según criterios técnicos, éticos y disciplinares. Este artículo analiza las particularidades de la publicación de datos en educación y ciencias sociales, y explora el papel de los catálogos de repositorios en la toma de decisiones informadas, con el objetivo de fortalecer la cultura de datos abiertos y contribuir al avance de la ciencia abierta en estas disciplinas.\\

\section{Método}

Este estudio se orienta a responder la pregunta de investigación: \textbf{¿qué características de un repositorio de datos de investigación son valoradas por los investigadores educativos al momento de publicar sus colecciones de datos?}. Para abordar esta interrogante, se diseñó una metodología mixta que integra enfoques cualitativos y cuantitativos, permitiendo una comprensión integral del fenómeno estudiado.\\

\subsection{Revisión Bibliográfica}
Se realizó una revisión de la literatura para identificar las necesidades y preferencias de los investigadores educativos en relación con la publicación de datos. Esta revisión se centró en estudios que abordan las barreras para el intercambio de datos, las estrategias propuestas para superarlas y los factores que influyen en la elección de repositorios por parte de los investigadores.\\
La revisión bibliográfica incluyó la búsqueda y el análisis de artículos científicos, informes técnicos, libros y otras publicaciones relevantes en bases de datos académicas, bibliotecas digitales y repositorios institucionales. Se utilizaron términos de búsqueda como ``repositorios de datos de investigación'', ``gestión de datos'', ``ciencia abierta'', ``principios FAIR'', ``metadatos'', ``intercambio de datos'', ``barreras para compartir datos'' y ``criterios de selección de repositorios''.\\

Estudios como el de \cite{barczak2022} señalan que, si bien la mayoría de los investigadores se muestran abiertos a compartir sus datos cuando se les solicita, el intercambio de datos no es una práctica generalizada. \cite{barczak2022} proponen que, para abordar esta problemática, se deben establecer mecanismos institucionales, como incentivos para el intercambio de datos y políticas de revistas que requieran la divulgación de datos para las publicaciones. También sugieren brindar apoyo administrativo a los investigadores en la preparación de los datos.\\

Investigaciones centradas en el contexto educativo, como la de \cite{casali2022}, indican que existe un desconocimiento significativo sobre los repositorios de datos abiertos entre los investigadores del área. \cite{casali2022} también identifican una actitud positiva de los investigadores hacia la publicación de sus datos, pero señalan la necesidad de políticas nacionales y programas institucionales que promuevan esta práctica. Además, destacan barreras como el tiempo y el esfuerzo requeridos para compartir los datos, la falta de financiamiento para la estandarización de los datos y las restricciones relacionadas con la seguridad y la confidencialidad de los datos.\\

\subsection{Consulta a Expertos}

Para complementar y enriquecer la revisión bibliográfica, se realizó una consulta semiestructurada a un experto en gestión de repositorios de datos, con amplia experiencia en la creación y administración de repositorios institucionales. Esta entrevista permitió obtener una perspectiva práctica y actualizada sobre los requerimientos, desafíos y expectativas de los investigadores al publicar datos, así como sobre las funcionalidades y limitaciones de las plataformas existentes.
La modalidad semiestructurada facilitó la exploración profunda de temas relevantes, sin restringir la espontaneidad del diálogo, lo que contribuyó a validar y ampliar los hallazgos teóricos con evidencia empírica y experiencia directa.\\

\subsection{Análisis de Catálogos de Repositorios}

Se llevó a cabo un análisis cuantitativo comparativo de los principales catálogos globales de repositorios de datos, con el objetivo de evaluar las características técnicas, organizativas y funcionales que ofrecen estas plataformas.\\

Este análisis se basó en el trabajo del Data Repository Attributes Working Group\footnote{\url{https://www.rd-alliance.org/groups/data-repository-attributes-wg/}}, que detalla los atributos más relevantes para la evaluación de repositorios de datos de investigación.\\

Los catálogos analizados fueron:
\begin{itemize}
    \item \emph{OpenDOAR}: Directorio global de repositorios de acceso abierto que ofrece acceso a recursos y productos académicos, incluyendo repositorios de datos.
    \item \emph{ROAR}: Registro de repositorios de acceso abierto, con información recopilada de diversas fuentes, incluyendo OpenDOAR.
    \item \emph{OpenAIRE}: Servicio que interconecta varios catálogos y proporciona información sobre las relaciones entre repositorios, autores, instituciones y agencias de financiación.
    \item \emph{FAIRshar}: Portal web que describe estándares, bases de datos y políticas de datos, con un enfoque en los principios FAIR.
    \item \emph{Repository Finder}: Interfaz web de consulta para el PID Graph, que recopila información sobre recursos académicos y sus interconexiones.
    \item \emph{Re3data}: Registro global de repositorios de datos de investigación de diferentes disciplinas académicas, recomendado por DataCite y requerido por OpenAIRE.
\end{itemize}

El análisis se basó en los atributos definidos por \cite{witt_2024_11221855}, que incluyen aspectos como asignación de identificadores persistentes, soporte para metadatos, interoperabilidad, políticas de licenciamiento, sostenibilidad y cumplimiento normativo. Se evaluó también la disponibilidad de interfaces de programación (API) para la extracción automatizada de información, un factor clave para la integración y actualización continua de datos.\\

\subsection{Triangulación y Validación}
La combinación de estos tres métodos, revisión bibliográfica, consulta a experto y análisis cuantitativo de catálogos, permitió triangular la información, aumentando la validez y confiabilidad de los resultados. Este enfoque metodológico integral facilitó la identificación de criterios técnicos y prácticos relevantes para la selección de repositorios, alineados con las necesidades reales de los investigadores educativos y las mejores prácticas internacionales en gestión de datos.\\
En conjunto, esta metodología robusta garantiza que las conclusiones y recomendaciones derivadas del estudio estén fundamentadas en evidencia diversa, actualizada y pertinente, contribuyendo así a fortalecer la infraestructura y cultura de datos abiertos en la investigación educativa.\\

\section{Resultados}

Esta sección presenta los resultados obtenidos a partir de la revisión bibliográfica, la consulta a expertos y el análisis de catálogos de repositorios.

\subsection{Revisión Bibliográfica}

La revisión bibliográfica permitió identificar varios factores que influyen en la elección de repositorios por parte de los investigadores. Entre ellos, se destacan:
\begin{itemize}
    \item \textbf{Reputación del repositorio}: Los investigadores tienden a preferir repositorios con buena reputación y reconocimiento dentro de su comunidad científica.
    \item \textbf{Facilidad de uso}: La interfaz del repositorio debe ser intuitiva y fácil de navegar, tanto para depositar como para acceder a los datos.
    \item \textbf{Disponibilidad de metadatos}: La existencia de metadatos completos y bien estructurados es crucial para facilitar la búsqueda y el descubrimiento de los datos.
    \item \textbf{Cumplimiento de los principios FAIR}: Los repositorios que cumplen con los principios FAIR (Encontrables, Accesibles, Interoperables, Reutilizables) son altamente valorados, ya que garantizan la accesibilidad y la reutilización de los datos.
    \item \textbf{Servicios de preservación a largo plazo}: Los investigadores buscan repositorios que aseguren la preservación de sus datos a lo largo del tiempo, garantizando su disponibilidad futura.
    \item \textbf{Apoyo técnico}: Un buen soporte técnico, que incluya documentación clara y asistencia personalizada, es fundamental para facilitar el uso del repositorio.
\end{itemize}

Además, el análisis de la literatura sobre las tipologías de repositorios reveló que existen diferentes tipos (disciplinarios, institucionales y generales), cada uno con sus propias ventajas y desventajas para los investigadores de diferentes áreas.
En particular, el trabajo de \cite{Jiang2023GlobalRepo} destaca por su análisis exhaustivo de las pautas para seleccionar repositorios de datos científicos confiables. Este estudio subraya la importancia de:
\begin{itemize}
    \item \textbf{Fiabilidad}: El repositorio debe cumplir con estándares y principios establecidos, como los principios FAIR, y contar con una estructura de gobierno clara y políticas transparentes.
    \item \textbf{Cumplimiento de los principios FAIR}: El repositorio debe proporcionar metadatos claros y admitir la interoperabilidad entre diferentes formatos de datos.
    \item \textbf{Reputación y reconocimiento}: El repositorio debe gozar de buena reputación dentro de la comunidad científica y contar con el respaldo de organizaciones profesionales o instituciones académicas.
    \item \textbf{Infraestructura técnica}: El repositorio debe contar con una infraestructura sólida en términos de capacidad de almacenamiento, seguridad de los datos y protocolos de respaldo.
    \item \textbf{Documentación y soporte al usuario}: El repositorio debe proporcionar documentación completa y un buen soporte técnico para facilitar su uso.
    \item \textbf{Sostenibilidad}: El repositorio debe contar con un modelo de financiación a largo plazo y un plan para garantizar su funcionamiento continuo.
\end{itemize}

\subsection{Consulta a Experto}

Las conclusiones derivadas de la entrevista con el experto se basan en un estudio interno realizado por la Agencia Nacional de Investigación e Innovación (ANII). Este estudio, aunque no publicado, proporcionó información valiosa sobre las perspectivas de los investigadores en el contexto local.
El estudio de la ANII reveló que los investigadores tienden a percibirse principalmente como consumidores de datos, más que como productores. Esta percepción influye en sus expectativas y necesidades en relación con los repositorios de datos.
El experto entrevistado, basándose en los resultados del estudio, destacó tres características clave que los investigadores consideran esenciales al momento de publicar sus datos:
\begin{itemize}
    \item \textbf{Asignación de identificadores persistentes}: La asignación de identificadores únicos y permanentes, preferentemente DOI, es vista como un requisito fundamental, especialmente debido a que muchas revistas científicas los exigen para la evaluación y citación de artículos.
    \item \textbf{Sostenibilidad y perdurabilidad}: La capacidad del repositorio para garantizar la conservación de los datos a largo plazo es considerada crucial para asegurar su disponibilidad y reutilización futura.
    \item \textbf{Interoperabilidad}: La capacidad del repositorio para facilitar la integración y el intercambio de datos entre diferentes plataformas y sistemas es vista como esencial para maximizar el impacto y la reutilización de los datos.
\end{itemize}

\subsection{Análisis de Catálogos de Repositorios}

\begin{table}[h]
    % \centering
    \begin{tabular}{p{1.5cm} | p{4cm} | p{3cm} | p{2.5cm} | p{2cm}}
    \hline
    Nombre & Descripción & Proceso de registro & Especialización & URL \\ \hline
    ROAR & Registro de repositorios de acceso abierto, con información recopilada de diversas fuentes, incluyendo OpenDOAR. & Revisión manual por comité de expertos & Repositorios institucionales de acceso abierto & \url{http://roar.eprints.org/} \\ \hline
    Open DOAR & Directorio global de repositorios de investigación de acceso abierto que ofrece acceso a recursos y productos académicos, incluyendo repositorios de datos & Revisión manual por comité de expertos & Repositorios institucionales de investigación de acceso abierto & \url{https://v2.sherpa.ac.uk/opendoar/} \\ \hline
    Re3data & Registro global de repositorios de datos de investigación de diferentes disciplinas académicas, recomendado por DataCite y requerido por OpenAIRE. Presenta un esquema de datos compatible con estándares comunitarios & Revisión manual por comité de expertos & Repositorios de datos de investigación & \url{https://www.re3data.org/} \\ \hline
    FAIR Sharing & Portal web que describe estándares, bases de datos y políticas de datos, con un enfoque en los principios FAIR. Complementa su información con la de openAIRE & Sistema de registro y revisión colaborativo y comunitario & Relaciona diferentes elementos de la ciencia abierta, dentro de ellos, los repositorios de datos & \url{https://fairsharing.org/} \\ \hline
    Open AIRE & Servicio que interconecta varios catálogos y proporciona información sobre las relaciones entre repositorios, autores, instituciones y agencias de financiación, aunque no registra información propia. La información primaria sobre repositorios es en su mayoría de re3data & Consume información de servicios externos, no registra información propia & Relación entre diferentes elementos de la ciencia abierta, poco enfocado en los repositorios & \url{https://graph.openaire.eu/docs/} \\ \hline
    Repository Finder & Interfaz web de consulta para el PID Graph (modelo de relacionamiento entre la infraestructura de la ciencia), que recopila información sobre recursos académicos y sus interconexiones. Promueve la interconexión de distintos recursos mediante la utilización del estándar de identificadores DOI & Repositorios de datos de investigación, con foco en los contenidos de los mismos & Permite el registro de organizaciones propias mediante un sistema relacionado DataCite Fábrica pero tambien indexa los datos de re3data & \url{https://commons.datacite.org/repositories} \\ \hline
    \end{tabular}
    \caption{\label{tab:catalogs}Tabla de catálogos de repositorios de datos.}
\end{table}

El análisis comparativo de los catálogos de repositories, cuyos detalles se presentan en la tabla~\ref{tab:catalogs}, permitió obtener una visión general de las características y funcionalidades que ofrecen estas plataformas. Los principales hallazgos fueron:
\begin{itemize}
    \item \textbf{re3data} se identificó como la fuente de datos primarios más completa y actualizada sobre repositorios de datos de investigación. Sin embargo, una pequeña porción de sus datos almacenados presentan importantes problemas de calidad que no permiten evaluar la vigencia o actualidad de los datos provistos, mientras que otra porción mayor presenta algunos errores de normalización subsanables automáticamente.
    \item \textbf{FAIRsharing} y \textbf{OpenAIRE} se destacaron como fuentes secundarias valiosas, ya que integran y relacionan información de diversas fuentes, proporcionando un contexto más amplio.
    \item \textbf{Repository Finder} ofrece información detallada sobre el contenido de cada repositorio, lo que puede ser útil para los investigadores que buscan datos específicos.
    \item \textbf{OpenDOAR} y \textbf{ROAR} no proporcionan datos con suficiente especificidad para un análisis automatizado en profundidad.
    \item La disponibilidad de una \textbf{API} se identificó como un factor crucial para facilitar la extracción eficiente de información de los repositorios.
\end{itemize}

\section{Conclusiones}

Este trabajo se desarrolló en el contexto de la investigación educativa y la publicación de datos de investigación asociados a esta área. Se reconoce que la investigación educativa comparte varias características con la investigación en general en lo que respecta a la gestión de datos, pero también presenta desafíos específicos relacionados con la heterogeneidad de los datos y las restricciones éticas y legales.\\

El análisis realizado subraya la importancia de abordar las necesidades de los investigadores en relación con los repositorios de datos. Se identificó que la interoperabilidad, el apoyo institucional y la asistencia técnica son aspectos cruciales para la comunidad investigadora. Además, se constató un desconocimiento generalizado sobre las opciones disponibles de repositorios de datos, a pesar de que los investigadores reconocen la importancia de las prácticas de publicación de datos.\\

Es importante destacar la apertura de los investigadores hacia la publicación de datos y su reconocimiento de los beneficios que esto conlleva tanto para quienes publican como para quienes acceden a los datos. Sin embargo, las herramientas tecnológicas actuales no parecen ofrecer una solución integral que cubra todas las necesidades de los investigadores educativos en materia de publicación de datos.\\

A pesar de esta limitación, el estudio permitió realizar una evaluación comparativa de los catálogos de repositorios de datos de investigación. En este sentido, \textbf{FAIRSharing} se destaca como la herramienta que ofrece los mejores niveles de curaduría y calidad de datos. También se resalta su modelo de sostenibilidad y la transparencia de sus criterios y procesos. \textbf{FAIRSharing} proporciona una visión orgánica del proceso de publicación y de la infraestructura de investigación abierta. Debido a que la base de datos de \textbf{FAIRSharing} carece de algunos registros, se recomienda realizar una consulta paralela en \textbf{Repository Finder}.\\

En cuanto a la evolución de las tecnologías para la catalogación de repositorios, \textbf{re3data}, si bien sigue cumpliendo un rol fundamental como fuente primaria de información, ha sido superado por herramientas que ofrecen una mejor usabilidad e interoperabilidad.\\

En conclusión, este estudio proporciona información valiosa para los investigadores educativos que buscan publicar sus datos de investigación. Al destacar los factores clave que influyen en la elección de repositorios y al evaluar las características de los catálogos de repositorios disponibles, se busca facilitar la toma de decisiones informadas y promover una mayor adopción de prácticas de datos abiertos en el campo de la investigación educativa.\\

\bibliographystyle{apalike}
\bibliography{sample}

\end{document}
